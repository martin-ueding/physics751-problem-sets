\documentclass[11pt, english, fleqn, DIV=15, headinclude, BCOR=1cm]{scrartcl}

\usepackage[
    bibatend,
    color,
]{../header}

\usepackage{booktabs}
\usepackage{pdflscape}

\usepackage{tikz}

\usepackage[tikz]{mdframed}
\newmdtheoremenv[%
    backgroundcolor=black!5,
    innertopmargin=\topskip,
    splittopskip=\topskip,
]{theorem}{Theorem}[section]

\newmdenv[%
    backgroundcolor=black!5,
    frametitlebackgroundcolor=black!10,
    roundcorner=5pt,
    skipabove=\topskip,
    innertopmargin=\topskip,
    splittopskip=\topskip,
    frametitle={Problem statement},
    frametitlerule=true,
]{problem}

\newmdenv[%
    backgroundcolor=white,
    frametitlebackgroundcolor=black!10,
    roundcorner=5pt,
    skipabove=\topskip,
    innertopmargin=\topskip,
    innerbottommargin=8cm,
    splittopskip=\topskip,
    frametitle={Side question},
    frametitlerule=true,
    %nobreak=true,
]{question}


\hypersetup{
    pdftitle=
}

\newcommand\inv{^{-1}}

\newcounter{totalpoints}
\newcommand\punkte[1]{#1\addtocounter{totalpoints}{#1}}

\newcounter{problemset}
\setcounter{problemset}{6}

\subject{physics751 -- Group Theory}
\ihead{physics751 -- Problem Set \arabic{problemset}}

\title{Problem Set \arabic{problemset}}

\publishers{Group 1 -- Patrick Matuschek}
\ofoot{Group 1 -- Patrick Matuschek}

\author{
    Martin Ueding \\ \small{\href{mailto:mu@martin-ueding.de}{mu@martin-ueding.de}}
}
\ifoot{Martin Ueding}

\ohead{\rightmark}

\begin{document}

\maketitle

\section{A representation of $D_3$ on polynomials}

\begin{problem}
    Consider the 6-dimensional space of polynomials of degree 2 in two
    variables $(x, y)$. Assuming that $x$ and $y$ transform under the dihedral
    group $D_3$ as the coordinates of a 2-vector with $\tens D^2(b)$, $\tens
    D^2(c)$ and $\tens D^2(c^2)$, we obtain a 6-dimensional representation of
    $D_3$ on the above function space. Write down the matrices of this
    representation …
\end{problem}

The vectors of the six dimensional vector space are the various $f(x, y)$. The
components of the vectors are the $a_i$. Addition of multiple vectors $f$ is
simply addition of the vectors $\vec a$. Exactly as I did on the last problem
set, I write everything in terms of vectors and matrices. The polynomial can be
written in components with the basis vectors $\ev$ out of $\{1, x, y, x^2, xy,
y^2\}$ like so:
\[
    f(x, y) = a^i \ev_i.
\]
Now I can find the six dimensional representation such that it fulfills this:
\[
    f\del{\tens D^2(g) \begin{pmatrix}
        x \\ y
    \end{pmatrix}} \overset!= \sbr{\tens D^6(g) \vec a}^i \ev_i.
\]

I will start with the first one:
\begin{align*}
    f\del{\tens D^2(c) \begin{pmatrix}
        x \\ y
    \end{pmatrix}}
    &= f(x, -y) \\
    &= a_1 + a_2 x - a_3 y + a_4 x^2 - a_5 xy + a_6 y^2 \\
    \intertext{%
        Now I can write this as a scalar product of a transformed vector:
    }
    &=
    \begin{pmatrix}
        a_1 \\ a_2 \\ - a_3 \\ a_4 \\ - a_5 \\ a_6
    \end{pmatrix}
    \begin{pmatrix}
        1 \\ x \\ y \\ x^2 \\ xy \\ y^2
    \end{pmatrix}. \\
    \intertext{%
        The matrix that has transformed $\vec a$ can be read off. To simplify,
        I omit all the zeros in the matrices.
    }
    &=
    \sbr{
        \begin{pmatrix}
            1 & & & & & \\
            & 1 & & & & \\
            & & -1 & & & \\
            & & & 1 & & \\
            & & & & -1 & \\
            & & & & & 1 \\
        \end{pmatrix}
        \begin{pmatrix}
            a_1 \\ a_2 \\ a_3 \\ a_4 \\ a_5 \\ a_6
        \end{pmatrix}
    }
    \begin{pmatrix}
        1 \\ x \\ y \\ x^2 \\ xy \\ y^2
    \end{pmatrix}
\end{align*}
From this, I can recover:
\[
    D^6(c) =
    \begin{pmatrix}
        1 & & & & & \\
          & 1 & & & & \\
          & & -1 & & & \\
          & & & 1 & & \\
          & & & & -1 & \\
          & & & & & 1 \\
    \end{pmatrix}.
\]

The next one is very similar, except that there are more terms.
\begin{align*}
    f\del{\tens D^2(c) \begin{pmatrix}
        x \\ y
    \end{pmatrix}}
    &= f\del{- \frac12 x - \frac{\sqrt3}2 y, \frac{\sqrt 3}2 x - \frac12 d} \\
    \intertext{%
        Expanding this gives a lot of terms:
    }
    &= a_1 + a_2 \sbr{- \frac12 x - \frac{\sqrt 3}{2} y} + a_3 \sbr{\frac{\sqrt
    3}2 x - \frac 12 y} + a_4 \sbr{- \frac 12 x - \frac{\sqrt 3}2 y}^2 \\&\qquad + a_5 \sbr{-
    \frac12 x - \frac{\sqrt 3}2 y} \sbr{\frac{\sqrt 3}2 x - \frac12 y} + a_6
    \sbr{\frac{\sqrt{3}}2 x - \frac 12 y}^2. \\
    \intertext{%
        The next step is to expand all the terms and regroup them with respect
        to the basis vectors. I omit typesetting the expansion here and skip
        right to the regrouping.
    }
    &= a_1 + x\sbr{- \frac12 a_2 + \frac{\sqrt 3}2 a_3} + y \sbr{-\frac{\sqrt
    3}2 a_2 - \frac12 a_3} + x^2 \sbr{\frac14 a_4 - \frac{\sqrt 3}4 a_5 + \frac 34
    a_6} \\&\qquad + xy \sbr{\frac32 a_4 - \frac12 a_5 - \frac{\sqrt 3}2 a_6} + y^2
    \sbr{\frac 34 a_4 + \frac{\sqrt 3}4 a_5 + \frac 14 a_6}
\end{align*}
The matrix can be read off here rather directly. I obtain
\[
    \tens D^6(c) =
    \begin{pmatrix}
        1 & & & & & \\
          & -\frac12 & \frac{\sqrt 3}2 & & & \\
          & - \frac{\sqrt 3}2 & -\frac12 & & & \\
          & & & \frac 14 & - \frac{\sqrt{3}}4 & \frac34 \\
          & & & \frac34 & - \frac12 & - \frac{\sqrt 3}2 \\
          & & & \frac34 & \frac{\sqrt 3}4 & \frac 14
    \end{pmatrix}.
\]

The same can be done for the group element $c^2$, were I yield
\[
    \tens D^6(c^2) =
    \begin{pmatrix}
        1 & & & & & \\
          & -\frac12 & -\frac{\sqrt 3}2 & & & \\
          & \frac{\sqrt 3}2 & \frac12 & & & \\
          & & & \frac 14 & \frac{\sqrt{3}}4 & \frac34 \\
          & & & -\frac34 & -1 & - \frac{\sqrt 3}2 \\
          & & & \frac34 & \frac{\sqrt 3}4 & \frac 14
    \end{pmatrix}.
\]

To be thorough, one should now test $[\tens D^6(b)]^2 = \mathbb 1_6$ and
$[\tens D^6(c)]^2 = \tens D^6(c^2)$ since a representation must behave like the
group elements.

\begin{problem}
    … and identify the invariant subspaces under $D_3$ and the corresponding
    irreducible representations.
\end{problem}

There is a $\R^1$ subspace which does not do anything. In this concrete case it
is just a constant summand on the functions, which is invariant under any
rotation or reflection.

Then there is the two dimensional representation which seems to be the
transpose of the $\tens D^6$. The fact that it is transposed seems a little
strange, maybe I did something wrong in my derivation? The fact that they are
almost equal is probably not too surprising, since there is only one
irreducible representation with dimension two in this representation.

The last one is a three dimensional representation and it seems rather
irreducible as well since the submatrix is dense.

The three irreducible representation are the trivial one, the $\tens D^2$ that
was given on the problem set transposed and the new three dimensional one that
is in the bottom right corner of all the three matrices that I have derived.

\end{document}

% vim: spell spelllang=en tw=79
