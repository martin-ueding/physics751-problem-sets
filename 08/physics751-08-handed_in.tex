\documentclass[11pt, english, fleqn, DIV=15, headinclude, BCOR=1cm]{scrartcl}

\usepackage[
    bibatend,
    color,
]{../header}

\usepackage{booktabs}
\usepackage{pdflscape}

\usepackage{tikz}

\usepackage[tikz]{mdframed}
\newmdtheoremenv[%
    backgroundcolor=white,
    innertopmargin=\topskip,
    splittopskip=\topskip,
]{theorem}{Theorem}[section]

\newmdenv[%
    backgroundcolor=SframeBackground,
    frametitlebackgroundcolor=SframeTitleBackground,
    roundcorner=5pt,
    skipabove=\topskip,
    innertopmargin=\topskip,
    splittopskip=\topskip,
    frametitle={Problem statement},
    frametitlerule=true,
]{problem}

\newmdenv[%
    backgroundcolor=white,
    frametitlebackgroundcolor=SframeTitleBackground,
    roundcorner=5pt,
    skipabove=\topskip,
    innertopmargin=\topskip,
    innerbottommargin=8cm,
    splittopskip=\topskip,
    frametitle={Side question},
    frametitlerule=true,
    %nobreak=true,
]{question}

\DeclareMathOperator{\Tr}{Tr}


\hypersetup{
    pdftitle=
}

\newcommand\inv{^{-1}}

\newcounter{totalpoints}
\newcommand\punkte[1]{#1\addtocounter{totalpoints}{#1}}

\newcounter{problemset}
\setcounter{problemset}{8}

\subject{physics751 -- Group Theory}
\ihead{physics751 -- Problem Set \arabic{problemset}}

\title{Problem Set \arabic{problemset}}

\publishers{Group 1 -- Patrick Matuschek}
\ofoot{Group 1 -- Patrick Matuschek}

\author{
    Martin Ueding \\ \small{\href{mailto:mu@martin-ueding.de}{mu@martin-ueding.de}}
}
\ifoot{Martin Ueding}

\ohead{\rightmark}

\begin{document}

\maketitle

\section{The character table of $S_4$}

\subsection{Parity}

\begin{problem}
    Identify the two 1-dim.\ representations associated with the parity of the
    elements of $S_4$.
\end{problem}

The first one dimensional representation ($\Gamma_1$) is the trivial one.
Everything gets mapped to 1. The second one ($\Gamma_2$) could be the
representation that maps to 1 and $-1$ depending on the parity. This means that
\begin{align*}
    D^{\Gamma_2}(K_1)& = 1 \\
    D^{\Gamma_2}(K_2)& = -1 \\
    D^{\Gamma_2}(K_3)& = 1 \\
    D^{\Gamma_2}(K_4)& = 1 \\
    D^{\Gamma_2}(K_5)& = -1 \\
\end{align*}

\subsection{Real characters}

\begin{problem}
    Show that the characters of all irreducible unitary representations of
    $S_n$ are real.
\end{problem}

\begin{proof}
    The inverse of any permutation in $S_n$ has the same cycle structure as the
    permutation itself. All elements of the same cycle structure are in the
    same conjugacy class. Each conjugacy class has one character, which is the
    same for every contained element. Therefore, $g$ and $g\inv$ have the same
    character. This means
    \begin{align*}
        \Tr\del{\tens D(g)} &= \Tr\del{\tens D(g\inv)}. \\
        \intertext{%
            The representation is taken to be unitary. This means that we can
            write the representation matrix of the inverse as the hermitian
            adjoint.
        }
        \Tr\del{\tens D(g)} &= \Tr\del{\tens D(g)^\dagger} \\
        \intertext{%
            The trace of the adjoint is just complex conjugated.
        }
        \Tr\del{\tens D(g)} &= \Tr\del{\tens D(g)}^* \\
        \intertext{%
            Both sides feature the same trace, the character of $g$.
        }
        \chi(g) &= \chi(g)^*
    \end{align*}
    This means that the characters are real.
\end{proof}

\subsection{Dimensions}

There are five conjugacy classes, therefore there are five irreducible
representations. The order of the group is 24. The only partition I could come
up with is
\[
    24 = 1 + 1 + 2^2 + 3^3 + 3^3.
\]

\subsection{Character of $K_3$}

The products are
\begin{gather*}
    (12)(34)(12)(34) = e \\
    (12)(34)(13)(24) = (14)(23) \\
    (12)(34)(14)(23) = (13)(24) \\
    (13)(24)(12)(34) = (14)(23) \\
    (13)(24)(13)(24) = e \\
    (13)(24)(14)(23) = (12)(34) \\
    (14)(23)(12)(34) = (13)(24) \\
    (14)(23)(13)(24) = (12)(34) \\ 
    (14)(23)(14)(23) = e.
\end{gather*}
In total, the product is
\[
    K_3 K_3 = 3 K_1 + 2 K_3.
\]
Therefore
\[
    c_{33}^1 = 3
    \eqnsep
    c_{33}^3 = 2.
\]

\begin{problem}
    Hint: Use
    \[
        |K_i| |K_j| \chi^{(\nu)}(K_i) \chi^{(\nu)}(K_j)
        = d_\nu \sum_k c_{ij}^k D^{(\nu)}(K_k)
    \]
    from your lecture.
\end{problem}

The formula that is given in the hint seems to have scalars on the left side
and the matrices (the representation $D^{(\nu)}(K_k)$) on the right. To me,
having the character on the right side as well makes more sense.

I found \parencite[(4.64)]{Luedeling/physics751-notes}, which is
\[
    n_a n_b \chi_{(\mu)}^a \chi_{(\mu)}^b = \chi_{(\mu)}^1 \sum_c c_{abc} n_c
    \chi_{(\mu)}^c.
\]
So $D^{(\nu)}(K_k)$ needs to be replaced by $|K_k| \chi^{(\nu)}(K_k)$ in the
notation from the exercise sheet.

Setting $i = j
= 3$ into the equation, this gives me
\[
    \chi^{\Gamma_3}_3 = \frac{10}{9}
    \eqnsep
    \chi^{\Gamma_3}_4 = \frac{5}{3}
    \eqnsep
    \chi^{\Gamma_3}_5 = \frac{5}{3}.
\]

The following table summarizes everything I have learned about this group so
far:

\begin{tabular}{c|ccccc}
    & $\chi_1$ & $\chi_2$ & $\chi_3$ & $\chi_4$ & $\chi_5$ \\
    \midrule
    $\Gamma_1$ & 1 & 1 & 1 & 1 & 1 \\
    $\Gamma_2$ & 1 & $-1$ & 1 & 1 & $-1$ \\
    $\Gamma_3$ & 2 & & $10/9$ & & \\
    $\Gamma_4$ & 3 & & $5/3$ & & \\
    $\Gamma_5$ & 3 & & $5/3$ & & \\
\end{tabular}

\subsection{Complete character table}

The product of $K_3$ and $K_4$ is a bit lengthy:
\begin{gather*}
    (12)(34) (123) = (243) \\
    (12)(34) (132) = (143) \\
    (12)(34) (124) = (234) \\
    (12)(34) (142) = (134) \\
    (12)(34) (134) = (142) \\
    (12)(34) (143) = (132) \\
    (12)(34) (234) = (124) \\
    (12)(34) (243) = (123) \\
    %
    (13)(24) (123) = (142) \\
    (13)(24) (132) = (234) \\
    (13)(24) (124) = (143) \\
    (13)(24) (142) = (123) \\
    (13)(24) (134) = (243) \\
    (13)(24) (143) = (124) \\
    (13)(24) (234) = (132) \\
    (13)(24) (243) = (134) \\
    %
    (14)(23) (123) = (134) \\
    (14)(23) (132) = (124) \\
    (14)(23) (124) = (132) \\
    (14)(23) (142) = (243) \\
    (14)(23) (134) = (123) \\
    (14)(23) (143) = (234) \\
    (14)(23) (234) = (143) \\
    (14)(23) (243) = (142)
\end{gather*}
When all elements are taken together, this is just
\[
    K_3 K_4 = 3 K_4.
\]
This means that the class constants are just $c_{34}^3 = 3$, all other
$c_{34}^k = 0$. With
\begin{align*}
    |K_3| |K_4| \chi^{\Gamma_\nu}_3 \chi^{\Gamma_\nu}_4 &= d_\nu c_{34}^3
    \chi^{\Gamma_\nu}_3 \\
    \iff 4 \cdot 8 \chi^{\Gamma_\nu}_3 \chi^{\Gamma_\nu}_4 &= d_\nu 3 \chi^{\Gamma_\nu}_3 \\
    \iff \chi^{\Gamma_\nu}_4 &= d_\nu \frac{3}{32}
\end{align*}

This lets me fill the character table with more elements. I have included the
number of elements in each conjugacy class so aid the application of the
orthogonality theorem.

\begin{tabular}{c|ccccc}
    & $\chi_1$ & $\chi_2$ & $\chi_3$ & $\chi_4$ & $\chi_5$ \\
    \midrule
    $|K|$ & 1 & 6 & 3 & 8 & 6 \\
    \midrule
    $\Gamma_1$ & 1 & 1 & 1 & 1 & 1 \\
    $\Gamma_2$ & 1 & $-1$ & 1 & 1 & $-1$ \\
    $\Gamma_3$ & 2 & & $10/9$ & 3/16 & \\
    $\Gamma_4$ & 3 & & $5/3$ & 9/32 & \\
    $\Gamma_5$ & 3 & & $5/3$ & 9/32 & \\
\end{tabular}

The remaining entries should be computable by the orthogonality theorems.
However, the table is inconsistent as it is. The sum of $\chi_3$ squared is not
$4!/4$, which should be the result. The sum of $\chi_4$ squared should be
$4!/8$, but it is not. The computed characters must be wrong.

\end{document}

% vim: spell spelllang=en tw=79
