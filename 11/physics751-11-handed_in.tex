\documentclass[11pt, english, fleqn, DIV=15, headinclude, BCOR=1cm]{scrartcl}

\usepackage[
    bibatend,
    %color,
]{../header}

\usepackage{booktabs}
\usepackage{pdflscape}

\usepackage{tikz}

\usepackage[tikz]{mdframed}
\newmdtheoremenv[%
    backgroundcolor=white,
    innertopmargin=\topskip,
    splittopskip=\topskip,
]{theorem}{Theorem}[section]

\newmdenv[%
    backgroundcolor=SframeBackground,
    frametitlebackgroundcolor=SframeTitleBackground,
    roundcorner=5pt,
    skipabove=\topskip,
    innertopmargin=\topskip,
    splittopskip=\topskip,
    frametitle={Problem statement},
    frametitlerule=true,
]{problem}

\newmdenv[%
    backgroundcolor=white,
    frametitlebackgroundcolor=SframeTitleBackground,
    roundcorner=5pt,
    skipabove=\topskip,
    innertopmargin=\topskip,
    innerbottommargin=8cm,
    splittopskip=\topskip,
    frametitle={Side question},
    frametitlerule=true,
    %nobreak=true,
]{question}

\DeclareMathOperator{\Tr}{Tr}


\hypersetup{
    pdftitle=
}

\newcommand\inv{^{-1}}

\newcounter{totalpoints}
\newcommand\punkte[1]{#1\addtocounter{totalpoints}{#1}}

\newcounter{problemset}
\setcounter{problemset}{11}

\subject{physics751 -- Group Theory}
\ihead{physics751 -- Problem Set \arabic{problemset}}

\title{Problem Set \arabic{problemset}}

\publishers{Group 1 -- Patrick Matuschek}
\ofoot{Group 1 -- Patrick Matuschek}

\author{
    Martin Ueding \\ \small{\href{mailto:mu@martin-ueding.de}{mu@martin-ueding.de}}
}
\ifoot{Martin Ueding}

\ohead{\rightmark}

\begin{document}

\maketitle

\section{Characters}

$C_\piup$ is the rotation around the $z$-axis with the angle $\piup$. The $i$
is the group element that corresponds to inversion. The group element $\sigma$
is the mirror symmetry.

The character of the mirror symmetry is computed directly using the hint that
$\sigma = C_\piup \circ i$:
\begin{align*}
    P_\sigma Y_{lm}
    &= P_{C_\piup} P_i Y_{lm} \\
    \intertext{%
        The action of $P_i$ is already given on the problem set. I just use
        that.
    }
    &= [-1]^l P_{C_\piup} Y_{lm} \\
    \intertext{%
        The action of the rotation on a single (fixed $m$) spherical harmonic
        is given at the top of the page on the problem set.
    }
    &= [-1]^l \exp(-\iup m \piup) Y_{lm} \\
    \intertext{%
        That exponential function is just a $[-1]^m$.
    }
    &= [-1]^{l+m} Y_{lm}
\end{align*}
I think of $[-1]^{l+m}$ as a diagonal matrix with index $m$. The character of
it is the sum of all the elements:
\[
    \chi(\sigma)
    = [-1]^l \sum_{m=-l}^l [-1]^m
\]
Theorem: $\chi(\sigma) = 1$.
\begin{proof}
    I will use induction for this. For $l = 0$, the character is 1. Now the
    induction step:
    \begin{align*}
        \chi(\sigma)_{l+1}
        &= [-1]^{l+1} \sum_{m=-l-1}^{l+1} [-1]^m \\
        &= - [-1]^{l} \sbr{[-1]^{-l-1} + [-1]^{l+1} + \sum_{m=-l}^{l} [-1]^m} \\
        &= - [-1]^{-1} - [-1]^{2l+1} - [-1]^{l} \sum_{m=-l}^{l} [-1]^m \\
        &= 1 + 1 - [-1]^{l} \sum_{m=-l}^{l} [-1]^m \\
        &= 1 + 1 - \chi(\sigma)_l \\
        &= 1 + 1 - 1 \\
        &= 1
    \end{align*}
    By the principle of full induction, the theorem is proven.
\end{proof}

Now I will tend to the improper rotations (rotoreflections) $S_\alpha =
C_\alpha \circ \sigma$. I can use the same method as before.
\begin{align*}
    P_{S_\alpha} Y_{lm}
    &= P_{C_\alpha} P_\sigma Y_{lm} \\
    \intertext{%
        From the previous part, I know the action of $P_\sigma$.
    }
    &= [-1]^{l+m} P_{C_\alpha} Y_{lm} \\
    &= [-1]^{l+m} \exp(-\iup m \alpha) Y_{lm}
\end{align*}
The character is the trace over this construct:
\begin{align*}
    \chi(S_\alpha)
    &= [-1]^l \sum_{m=-l}^l [-1]^m \exp(-\iup m \alpha) \\
    \intertext{%
        I move the $[-1]^m$ into the exponential function.
    }
    &= [-1]^l \sum_{m=-l}^l \exp\del{-\iup m [\alpha+\piup]} \\
    \intertext{%
        This only differs by the factor in front from the character of
        $C_{\alpha+\piup}$.
    }
    &= [-1]^l \chi(C_{\alpha+\piup}) \\
    &= [-1]^l \frac{\sin\del{\sbr{l + \frac12}[\alpha +
    \piup]}}{\sin\del{\frac{\alpha+\piup}2}}
\end{align*}

I summarize this in a partial character table:

\begin{tabular}{c|cc}
    & $\sigma$ & $S_\alpha$ \\
    \midrule
    $\chi^{Y_{lm}}$ & $1$ & $\displaystyle
    [-1]^l \frac{\sin\del{\sbr{l + \frac12}[\alpha +
    \piup]}}{\sin\del{\frac{\alpha+\piup}2}}
    $
\end{tabular}

\section{Determine the character table}

The group $\mathrm{O_h}$ has ten conjugacy classes. The ones with rotations
around ${2\piup/3}\,\si{\radian}$ should have the same character, I think: The
character of a rotation was only dependant on its angle $\alpha$, and the
$z$-axis was used without loss of generality. Therefore there are only eight
characters that I have to compute. The classes are represented by
\[
    e
    \eqnsep
    C_{\piup/2}
    \eqnsep
    C_{2\piup/3}
    \eqnsep
    C_{\piup}
    \eqnsep
    i
    \eqnsep
    S_{\piup/2}
    \eqnsep
    S_{2\piup/3}
    \eqnsep
    S_{\piup}
\]

The basis has seven dimensions, therefore I need seven dimensional
representations.

The first characters are
\[
    \chi(e) = 2l+1 = 7
    \eqnsep
    \chi(i) = [-1]^3 [2l+1] = - 7.
\]
Then
\[
    \chi(C_{\piup/2})
    = \frac{\sin\del{\frac72 \frac\piup2}}{\sin\del{\frac\piup4}}
    = - 1.
\]
From this,
\[
    \chi(S_{\piup/2})
    = 1
\]
follows by the derivations in the previous exercise.

Then
\[
    \chi(C_{2\piup/3})
    = \frac{\sin\del{\frac72 \frac{2\piup}3}}{\sin\del{\frac\piup3}}
    = \frac{\sin\del{\frac73 \piup}}{\sin\del{\frac\piup3}}
    = \frac{\sin\del{\frac\piup3}}{\sin\del{\frac\piup3}}
    = 1
    \eqnsep
    \chi(S_{2\piup/3}) = -1.
\]
And
\[
    \chi(C_{\piup})
    = \frac{\sin\del{\frac72 \piup}}{\sin\del{\frac\piup2}}
    = - 1
    \eqnsep
    \chi(S_{\piup}) = 1.
\]

\begin{tabular}{c|*8S}
    & {$e$} & {$C_{\piup/2}$} & {$C_{2\piup/3}$} & {$C_{\piup}$} & {$i$} & {$S_{\piup/2}$} &
    {$S_{2\piup/3}$} & {$S_{\piup}$} \\
    \midrule
    $\chi^{Y_{3m}}$ & 7 & -1 & 1 & -1 & -7 & 1 & 2 & 1
\end{tabular}

\section{Reduced symmetry}

When the particle has angular momentum $J$, there is only rotational symmetry
around $\vec J$. The important thing probably is that only the representation
with $Y_{3m}$ is the one that applies to this particle. Therefore I have to see
how this decomposes into irreducible representations of the octahedral group.
The previous problems asked to derive some characters from the full
$\mathrm{SO}(3)$ group and for the octahedral group with $Y_{3m}$ as the basis.
Now I need the character table of $\mathrm{O_h}$ in order to find the linear
combinations that give me the row that I have derived.

From the lecture on 2015-01-07 and the review on 2015-01-14 I know that the
decomposition of a particle with a particular angular momentum is given by
$\mathrm E \oplus \mathrm T_2$. Angular momentum of $l=2$ is the quadrupole.
The quadrupole has decomposition $\mathrm{E_g} \oplus \mathrm{T_{2g}}$
\parencite{character_table_Oh}. With $l = 3$ in this case, is is the octopole
which is given by $\mathrm{A_{2u}} \oplus \mathrm{T_{1u}} \oplus
\mathrm{T_{2u}}$ \parencite{character_table_Oh}. It makes sense that a particle
with odd partity due to its angular momentum is represented by odd irreducible
representations.

The full character table of the octahedral group is shown in
table~\ref{tab:char_Oh}.

\begin{table}[htbp]
    \centering
\begin{tabular}{c|*{10}S}
    $\mathrm{O_h}$ & {$E$} & {$8 C_3$} & {$6 C_2$} &
    {$6 C_4$} & {$3 C_2 = (C_4)^2$} & {$i$} & {$6 S_4$} & {$8 S_6$} &
{$3 \sigma_h$} & {$6 \sigma_d$} \\
\midrule
$\mathrm{A_{1g}}$ & +1 & +1 & +1 & +1 & +1 & +1 & +1 & +1 & +1 & +1 \\
$\mathrm{A_{2g}}$ & +1 & +1 & -1 & -1 & +1 & +1 & -1 & +1 & +1 & -1 \\
$\mathrm{E_{g}}$ & +2 & -1 & 0 & 0 & +2 & +2 & 0 & -1 & +2 & 0 \\
$\mathrm{T_{1g}}$ & +3 & 0 & -1 & +1 & -1 & +3 & +1 & 0 & -1 & -1 \\
$\mathrm{T_{2g}}$ & +3 & 0 & +1 & -1 & -1 & +3 & -1 & 0 & -1 & +1 \\
$\mathrm{A_{1u}}$ & +1 & +1 & +1 & +1 & +1 & -1 & -1 & -1 & -1 & -1 \\
$\mathrm{A_{2u}}$ & +1 & +1 & -1 & -1 & +1 & -1 & +1 & -1 & -1 & +1 \\
$\mathrm{E_{u}}$ & +2 & -1 & 0 & 0 & +2 & -2 & 0 & +1 & -2 & 0 \\
$\mathrm{T_{1u}}$ & +3 & 0 & -1 & +1 & -1 & -3 & -1 & 0 & +1 & +1 \\
$\mathrm{T_{2u}}$ & +3 & 0 & +1 & -1 & -1 & -3 & +1 & 0 & +1 & -1
\end{tabular}
    \caption{%
        Character table of the octahedral group. \parencite{character_table_Oh}
    }
    \label{tab:char_Oh}
\end{table}

I will change the notations of the rotations a bit from the previous character
table.

\begin{tabular}{c|*8S}
    & {$e$} & {$C_4$} & {$C_3$} & {$C_2$} & {$i$} & {$S_{\piup/2}$} &
    {$S_{3\piup/2}$} & {$S_{\piup}$} \\
    \midrule
    $\chi^{Y_{3m}}$ & 7 & -1 & 1 & -1 & -7 & 1 & 2 & 1
\end{tabular}

The dimension has to match up, and the character of the inversion as well.
Therefore, only odd (“u”) can contribute. This was motivated earlier on with
the parity of the particle itself as well. That restricts to the lower half of
the character table. Now I tend to $C_2$. The only irreducible representation
that is left has a character of 2, which is too large. Taking $2 \mathrm{E_u}$
and one $\mathrm T_1$ will not match up. So it has to be one $\mathrm A$ and
two $\mathrm T$. Looking at $C_4$ with this in mind, only $\mathrm A_2$ is a
possible choice. Therefore it is
\[
    \mathrm{A_{2u}} \oplus \mathrm{T_{1u}} \oplus \mathrm{T_{2u}}.
\]

\section{Mass}

It is probably better to use the higher volume since energy differences seem to
scale with the inverse volume \parencite{luescher/volume_dependence}. I will
take the $L = 20$ values.

The entries that are on all of the three irreducible representations and around
the same energy value seem to be around \num{1.45}.

\end{document}

% vim: spell spelllang=en tw=79
