\documentclass[11pt, english, fleqn, DIV=15, headinclude, BCOR=1cm]{scrartcl}

\usepackage[
    bibatend,
    color,
]{../header}

\usepackage{booktabs}
\usepackage{pdflscape}

\usepackage{tikz}

\usepackage[tikz]{mdframed}
\newmdtheoremenv[%
    backgroundcolor=white,
    innertopmargin=\topskip,
    splittopskip=\topskip,
]{theorem}{Theorem}[section]

\newmdenv[%
    backgroundcolor=SframeBackground,
    frametitlebackgroundcolor=SframeTitleBackground,
    roundcorner=5pt,
    skipabove=\topskip,
    innertopmargin=\topskip,
    splittopskip=\topskip,
    frametitle={Problem statement},
    frametitlerule=true,
]{problem}

\newmdenv[%
    backgroundcolor=white,
    frametitlebackgroundcolor=SframeTitleBackground,
    roundcorner=5pt,
    skipabove=\topskip,
    innertopmargin=\topskip,
    innerbottommargin=8cm,
    splittopskip=\topskip,
    frametitle={Side question},
    frametitlerule=true,
    %nobreak=true,
]{question}

\DeclareMathOperator{\Tr}{Tr}


\hypersetup{
    pdftitle=
}

\newcommand\inv{^{-1}}

\newcounter{totalpoints}
\newcommand\punkte[1]{#1\addtocounter{totalpoints}{#1}}

\newcounter{problemset}
\setcounter{problemset}{11}

\subject{physics751 -- Group Theory}
\ihead{physics751 -- Problem Set \arabic{problemset}}

\title{Problem Set \arabic{problemset}}

\publishers{Group 1 -- Patrick Matuschek}
\ofoot{Group 1 -- Patrick Matuschek}

\author{
    Martin Ueding \\ \small{\href{mailto:mu@martin-ueding.de}{mu@martin-ueding.de}}
}
\ifoot{Martin Ueding}

\ohead{\rightmark}

\begin{document}

\maketitle

\section{Characters}

$C_\piup$ is the rotation around the $z$-axis with the angle $\piup$. The $i$
is the group element that corresponds to inversion. The group element $\sigma$
is the mirror symmetry.

The character of the mirror symmetry is computed directly using the hint that
$\sigma = C_\piup \circ i$:
\begin{align*}
    P_\sigma Y_{lm}
    &= P_{C_\piup} P_i Y_{lm} \\
    \intertext{%
        The action of $P_i$ is already given on the problem set. I just use
        that.
    }
    &= [-1]^l P_{C_\piup} Y_{lm} \\
    \intertext{%
        The action of the rotation on a single (fixed $m$) spherical harmonic
        is given at the top of the page on the problem set.
    }
    &= [-1]^l \exp(-\iup m \piup) Y_{lm} \\
    \intertext{%
        That exponential function is just a $[-1]^m$.
    }
    &= [-1]^{l+m} Y_{lm}
\end{align*}
I think of $[-1]^{l+m}$ as a diagonal matrix with index $m$. The character of
it is the sum of all the elements:
\[
    \chi(\sigma)
    = [-1]^l \sum_{m=-l}^l [-1]^m
\]
Theorem: $\chi(\sigma) = 1$.
\begin{proof}
    I will use induction for this. For $l = 0$, the character is 1. Now the
    induction step:
    \begin{align*}
        \chi(\sigma)_{l+1}
        &= [-1]^{l+1} \sum_{m=-l-1}^{l+1} [-1]^m \\
        &= - [-1]^{l} \sbr{[-1]^{-l-1} + [-1]^{l+1} + \sum_{m=-l}^{l} [-1]^m} \\
        &= - [-1]^{-1} - [-1]^{2l+1} - [-1]^{l} \sum_{m=-l}^{l} [-1]^m \\
        &= 1 + 1 - [-1]^{l} \sum_{m=-l}^{l} [-1]^m \\
        &= 1 + 1 - \chi(\sigma)_l \\
        &= 1 + 1 - 1 \\
        &= 1
    \end{align*}
    By the principle of full induction, the theorem is proven.
\end{proof}

Now I will tend to the improper rotations (rotoreflections) $S_\alpha =
C_\alpha \circ \sigma$. I can use the same method as before.
\begin{align*}
    P_{S_\alpha} Y_{lm}
    &= P_{C_\alpha} P_\sigma Y_{lm} \\
    \intertext{%
        From the previous part, I know the action of $P_\sigma$.
    }
    &= [-1]^{l+m} P_{C_\alpha} Y_{lm} \\
    &= [-1]^{l+m} \exp(-\iup m \alpha) Y_{lm}
\end{align*}
The character is the trace over this construct:
\begin{align*}
    \chi(S_\alpha)
    &= [-1]^l \sum_{m=-l}^l [-1]^m \exp(-\iup m \alpha) \\
    \intertext{%
        I move the $[-1]^m$ into the exponential function.
    }
    &= [-1]^l \sum_{m=-l}^l \exp\del{-\iup m [\alpha+\piup]} \\
    \intertext{%
        This only differs by the factor in front from the character of
        $C_{\alpha+\piup}$.
    }
    &= [-1]^l \chi(C_{\alpha+\piup}) \\
    &= [-1]^l \frac{\sin\del{\sbr{l + \frac12}[\alpha +
    \piup]}}{\sin\del{\frac{\alpha+\piup}2}} \\
\end{align*}

\section{Determine the character table}

The group $\mathrm{O_h}$ has ten conjugacy classes. The ones with rotations
around ${2\piup/3}\,\si{\radian}$ should have the same character, I think: The
character of a rotation was only dependant on its angle $\alpha$, and the
$z$-axis was used without loss of generality. Therefore there are only eight
characters that I have to compute. The classes are represented by
\[
    e
    \eqnsep
    C_{\piup/2}
    \eqnsep
    C_{3\piup/2}
    \eqnsep
    C_{\piup}
    \eqnsep
    i
    \eqnsep
    S_{\piup/2}
    \eqnsep
    S_{3\piup/2}
    \eqnsep
    S_{\piup}
\]

The basis has seven dimensions, therefore I need seven dimensional
representations.

The first characters are
\[
    \chi(e) = 2l+1 = 7
    \eqnsep
    \chi(i) = [-1]^3 [2l+1] = - 7.
\]
Then
\[
    \chi(C_{\piup/2})
    = \frac{\sin\del{\frac72 \frac\piup2}}{\sin\del{\frac\piup4}}
    = - 1.
\]
From this,
\[
    \chi(S_{\piup/2})
    = 1
\]
follows by the derivations in the previous exercise.

Then
\[
    \chi(C_{3\piup/2})
    = \frac{\sin\del{\frac72 \frac{3\piup}2}}{\sin\del{\frac\piup3}}
    = - \sqrt{\frac23}
    \eqnsep
    \chi(S_{3\piup/2}) = \sqrt{\frac23}.
\]
And
\[
    \chi(C_{\piup})
    = \frac{\sin\del{\frac72 \piup}}{\sin\del{\frac\piup2}}
    = - 1
    \eqnsep
    \chi(S_{\piup}) = 1.
\]


\section{Reduced symmetry}

When the particle has angular momentum $J$, there is only rotational symmetry
around $\vec J$.


\end{document}

% vim: spell spelllang=en tw=79
