\documentclass[11pt, english, fleqn, DIV=15, headinclude, BCOR=1cm]{scrartcl}

\usepackage[
    bibatend,
    color,
]{../header}

\usepackage{booktabs}
\usepackage{pdflscape}

\usepackage{tikz}

\usepackage[tikz]{mdframed}
\newmdtheoremenv[%
    backgroundcolor=black!5,
    innertopmargin=\topskip,
    splittopskip=\topskip,
]{theorem}{Theorem}[section]

\newmdenv[%
    backgroundcolor=black!5,
    frametitlebackgroundcolor=black!10,
    roundcorner=5pt,
    skipabove=\topskip,
    innertopmargin=\topskip,
    splittopskip=\topskip,
    frametitle={Problem Statement},
    frametitlerule=true,
]{problem}

\newmdenv[%
    backgroundcolor=black!5,
    frametitlebackgroundcolor=black!10,
    roundcorner=5pt,
    skipabove=\topskip,
    innertopmargin=\topskip,
    splittopskip=\topskip,
    frametitle={Side Question},
    frametitlerule=true,
]{question}


\hypersetup{
    pdftitle=
}

\newcommand\inv{^{-1}}

\newcounter{totalpoints}
\newcommand\punkte[1]{#1\addtocounter{totalpoints}{#1}}

\newcounter{problemset}
\setcounter{problemset}{3}

\subject{physics751 -- Group Theory}
\ihead{physics751 -- Problem Set \arabic{problemset}}

\title{Problem Set \arabic{problemset}}

\publishers{Group 1 -- Patrick Matuschek}
\ofoot{Group 1 -- Patrick Matuschek}

\author{
    Martin Ueding \\ \small{\href{mailto:mu@martin-ueding.de}{mu@martin-ueding.de}}
}
\ifoot{Martin Ueding}

\ohead{\rightmark}

\begin{document}

\maketitle

\section{Pauli matrices}

\newcommand\1{\sigma_1}
\newcommand\2{\sigma_2}
\newcommand\3{\sigma_3}
\newcommand\ssep{,\;}

\subsection{Formation of a group}

\begin{problem}
    Show that all possible products generated by $\1$ and $\2$ form a group.
\end{problem}

The identity element of this group is the $\mathbb 1_2$ identity matrix.

To obtain the group structure, I need to find out how the products of the
elements behave.

The first things to notice is
\[
    \1\1 = e
    \eqnsep
    \2\2 = e.
\]
Then
\[
    \1\2 = \iup \3
    \eqnsep
    \2\1 = - \iup \3.
\]
Those can be used to yield
\[
    \1\2\1 = \iup \3 \1 = - \2
    \eqnsep
    \2\1\2 = - \iup \3 \2 = \iup \2\3 = - \1.
\]
The last element needed is
\[
    \1\2\1\2 = - \3\3 = - e
    \eqnsep
    \2\1\2\1 = - \3\3 = - e.
\]

There is only so much staggering with the group elements. (Infinite) products
of $\1$ and $\2$ will come down to at most 7 factors. When two equal factors
are next to each other, they reduce to the identity, and $[\1\2]^4$ is the
identity as well. This means closure.

Then there are inverses for any group element because $\1$ and $\2$ are their
own inverses. The inverse of a product is just the reversed order of elements.
Whatever way the element and its inverse is combined, there will be squares of
$\1$ and $\2$ which are the identity, until the whole expression reduced to
unity.

The other way is to work with the representation in form of the matrices.
Matrices have a multiplication law that is non-commutative and associative. The
group is supposed to be made up of all possible products of $\1$ and $\2$, so
by that definition closure is baked in.

Another way to see that it forms a group is that there exists a multiplication
table for it (see below), which implies closure and the existence of an inverse
since there is an identity in each row and column.

\subsection{Multiplication table and orders}

\begin{problem}
    Write down the multiplication table of this group and identify its order,
    the order of each element, the classes, the invariant subgroups and their
    quotient groups.
\end{problem}

\subsubsection{Multiplication table}

There are two ways now: Building up everything out of the generators $\1$ and
$\2$ as well as the identity $e$ or introducing new names for some of the
products.

The variant in table~\ref{tab:generators} uses only the generators. And in
table~\ref{tab:symbols} I have introduced $-e := \1\2\1\2$.

\begin{table}[htbp]
    \centering
    \begin{tabular}{c|ccccccc}
        $e$ & $\1$ & $\2$ & $\1\2$ & $\2\1$ & $-\1$ & $-\2$ & $-e$ \\
        \midrule
        $\1$ & $e$ & $\1\2$ & $\2$ & $-\2$ & $-e$ & $-e$ & $-\1$ \\
        $\2$ & $\2\1$ & $e$ & $-\1$ & $\1$ & $\1\2$ & $-e$ & $-\2$ \\
        $\1\2$ & $-\2$ & $\1$ & $-e$ & $e$ & $\2$ & $-\1$ & $\2\1$ \\
        $\2\1$ & $\2$ & $-\1$ & $e$ & $-e$ & $-\2$ & $\1$ & $\1\2$ \\
        $-\1$ & $-e$ & $\2\1$ & $-\2$ & $\2$ & $e$ & $\1\2$ & $\1$ \\
        $-\2$ & $\1\2$ & $-e$ & $\1$ & $-\1$ & $\2\1$ & $e$ & $\2$ \\
        $-e$ & $-\1$ & $-\2$ & $\2\1$ & $\1\2$ & $\1$ & $\2$ & $e$ \\
    \end{tabular}
    \caption{%
        Multiplication table with $-e := \1\2\1\2$ introduced.
    }
    \label{tab:symbols}
\end{table}

\begin{landscape}
    \begin{table}[p]
        \centering
        \begin{tabular}{c|ccccccc}
            $e$ & $\1$ & $\2$ & $\1\2$ & $\2\1$ & $\2\1\2$ & $\1\2\1$ & $\1\2\1\2$ \\
            \midrule
            $\1$ & $e$ & $\1\2$ & $\2$ & $\1\2\1$ & $\1\2\1\2$ & $\1\2\1\2$ & $\2\1\2$ \\
            $\2$ & $\2\1$ & $e$ & $\2\1\2$ & $\1$ & $\1\2$ & $\1\2\1\2$ & $\1\2\1$ \\
            $\1\2$ & $\1\2\1$ & $\1$ & $\1\2\1\2$ & $e$ & $\2$ & $\2\1\2$ & $\2\1$ \\
            $\2\1$ & $\2$ & $\2\1\2$ & $e$ & $\1\2\1\2$ & $\1\2\1$ & $\1$ & $\1\2$ \\
            $\2\1\2$ & $\1\2\1\2$ & $\2\1$ & $\1\2\1$ & $\2$ & $e$ & $\1\2$ & $\1$ \\
            $\1\2\1$ & $\1\2$ & $\1\2\1\2$ & $\1$ & $\2\1\2$ & $\2\1$ & $e$ & $\2$ \\
            $\1\2\1\2$ & $\2\1\2$ & $\1\2\1$ & $\2\1$ & $\1\2$ & $\1$ & $\2$ & $e$ \\
        \end{tabular}
        \caption{%
            Multiplication table built up from the generators only.
        }
        \label{tab:generators}
    \end{table}
\end{landscape}

\subsubsection{Order of the group}

There are 8 elements in the whole thing:
\[
    \{ e\ssep \1\ssep \2\ssep \1\2\ssep \2\1\ssep \2\1\2\ssep \1\2\1\ssep \1\2\1\2 \}.
\]
Therefore, the order is 8.

\subsubsection{Orders of the elements}

These are the powers needed to get each element back to unity:
\begin{gather*}
    \1^2 = e \\
    \2^2 = e \\
    [\1\2]^4 = [\iup \3]^4 = [-e]^2 = e \\
    [\2\1]^4 = [-\iup \3]^4 = [-e]^2 = e \\
    [\1\2\1]^2 = \1\2\1 \1\2\1 = e \\
    [\2\1\2]^2 = \2\1\2 \2\1\2 = e \\
    [\1\2\1\2]^2 = [\1\2\1\2]^4 = e
\end{gather*}

The orders are written down in table~\ref{tab:orders}.

\begin{table}[htbp]
    \centering
    \begin{tabular}{cc}
        Element & Order \\
        \midrule
        $\1$ & 2 \\
        $\2$ & 2 \\
        $\1\2$ & 4 \\
        $\2\1$ & 4 \\
        $\1\2\1$ & 2 \\
        $\2\1\2$ & 2 \\
        $\1\2\1\2$ & 2
    \end{tabular}
    \caption{%
        Orders of the elements.
    }
    \label{tab:orders}
\end{table}

\subsubsection{Classes}

I assume “classes” means “conjugacy classes”?

The definition of conjugacy of $a, b \in G$ is
\[
    a \sim b
    \iff
    \exists g \in G \colon
    a = g b g\inv.
\]

What I did is to go through the elements $b \in G$ and iterate through the $g
\in G$ and see which $a$ I would get. These are the results:
\begin{gather*}
    [e] = \{ e \} \\
    [\1] = \{ \1 \ssep \2\1\2 \} \\
    [\2] = \{ \2 \ssep \1\2\1 \} \\
    [\1\2] = \{ \1\2 \ssep \2\1 \} \\
    [\1\2\1\2] = \{ \1\2\2\1 \}
\end{gather*}

\subsubsection{Invariant subgroup}

Invariant subgroups are made from conjugacy classes. $[e]$ has to be in it in
any case. I tried to start with $[\1]$ or $[\2]$, but the groups were either
not normal, no groups at all or the whole group itself. So I started with the
other two and got a normal subgroup:
\[
    H := \{ e \ssep \1\2 \ssep \2\1 \ssep \1\2\1\2 \}.
\]

Then I tried to pre and postmultiply it with $\1$ and got the same cosets. By
analogy, this works with $\2$ as well. Doing the same with $\1\2$ and $\2\1$ is
trivial since they are contained in the group, right?

\subsubsection{Quotient group}

\subsection{Isomorphicity}

\begin{problem}
    Prove that this group is isomorphic to $D_4$.
\end{problem}

Given the rotation $c$ and the reflection $b$ the elements of $D_4$ are:
\[
    D_4 = \{ e\ssep c\ssep c^2\ssep c^3\ssep b\ssep bc\ssep bc^2\ssep bc^3 \}.
\]
Those are 8 elements in total, meaning that the isomorphicity is not ruled out
yet.

\end{document}

% vim: spell spelllang=en tw=79
