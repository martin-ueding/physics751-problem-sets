\documentclass[11pt, english, fleqn, DIV=15, headinclude, BCOR=1cm]{scrartcl}

\usepackage[bibatend]{../header}

\usepackage{booktabs}
\usepackage{pdflscape}

\usepackage{tikz}

\usepackage[tikz]{mdframed}
\newmdtheoremenv[%
    backgroundcolor=black!5,
    innertopmargin=\topskip,
    splittopskip=\topskip,
]{theorem}{Theorem}[section]

\newmdenv[%
    backgroundcolor=black!5,
    frametitlebackgroundcolor=black!10,
    roundcorner=5pt,
    innertopmargin=\topskip,
    splittopskip=\topskip,
    frametitle={Problem Statement},
    frametitlerule=true,
]{problem}

\hypersetup{
    pdftitle=
}

\newcommand\inv{^{-1}}

\newcounter{totalpoints}
\newcommand\punkte[1]{#1\addtocounter{totalpoints}{#1}}

\newcounter{problemset}
\setcounter{problemset}{3}

\subject{physics751 -- Group Theory}
\ihead{physics751 -- Problem Set \arabic{problemset}}

\title{Problem Set \arabic{problemset}}

\publishers{Group 1 -- Patrick Matuschek}
\ofoot{Group 1 -- Patrick Matuschek}

\author{
    Martin Ueding \\ \small{\href{mailto:mu@martin-ueding.de}{mu@martin-ueding.de}}
}
\ifoot{Martin Ueding}

\ohead{\rightmark}

\begin{document}

\maketitle

\section{Pauli Matrices}

\newcommand\1{\sigma_1}
\newcommand\2{\sigma_2}
\newcommand\3{\sigma_3}

\begin{problem}
    Show that all possible products generated by $\1$ and $\2$ form a group.
\end{problem}

To obtain the group structure, I need to find out how the products of the
elements behave.

The first things to notice is
\[
    \1\1 = e
    \eqnsep
    \2\2 = e.
\]
Then
\[
    \1\2 = \iup \3
    \eqnsep
    \2\1 = - \iup \3.
\]
Those can be used to yield
\[
    \1\2\1 = \iup \3 \1 = - \2
    \eqnsep
    \2\1\2 = - \iup \3 \2 = \iup \2\3 = - \1.
\]
The last element needed is
\[
    \1\2\1\2 = - \3\3 = - e
    \eqnsep
    \2\1\2\1 = - \3\3 = - e.
\]

There are two ways now: Building up everything out of the generators $\1$ and
$\2$ as well as the identity $e$ or introducing new names for some of the
products.

The variant in table~\ref{tab:generators} uses only the generators.

\begin{landscape}
    \begin{table}[p]
        \centering
        \begin{tabular}{c|ccccccc}
            $e$ & $\1$ & $\2$ & $\1\2$ & $\2\1$ & $\2\1\2$ & $\1\2\1$ & $\1\2\1\2$ \\
            \midrule
            $\1$ & $e$ & $\1\2$ & $\2$ & $\1\2\1$ & $\1\2\1\2$ & $\1\2\1\2$ & $\2\1\2$ \\
            $\2$ & $\2\1$ & $e$ & $\2\1\2$ & $\1$ & $\1\2$ & $\1\2\1\2$ & $\1\2\1$ \\
            $\1\2$ & $\1\2\1$ & $\1$ & $\1\2\1\2$ & $e$ & $\2$ & $\2\1\2$ & $\2\1$ \\
            $\2\1$ & $\2$ & $\2\1\2$ & $e$ & $\1\2\1\2$ & $\1\2\1$ & $\1$ & $\1\2$ \\
            $\2\1\2$ & $\1\2\1\2$ & $\2\1$ & $\1\2\1$ & $\2$ & $e$ & $\1\2$ & $\1$ \\
            $\1\2\1$ & $\1\2$ & $\1\2\1\2$ & $\1$ & $\2\1\2$ & $\2\1$ & $e$ & $\2$ \\
            $\1\2\1\2$ & $\2\1\2$ & $\1\2\1$ & $\2\1$ & $\1\2$ & $\1$ & $\2$ & $e$ \\
        \end{tabular}
        \caption{%
            Multiplication table built up from the generators only.
        }
        \label{tab:generators}
    \end{table}
\end{landscape}

    And here I have introduced $-e := \1\2\1\2$:

    \begin{tabular}{c|ccccccc}
        $e$ & $\1$ & $\2$ & $\1\2$ & $\2\1$ & $-\1$ & $-\2$ & $-e$ \\
        \midrule
        $\1$ & $e$ & $\1\2$ & $\2$ & $-\2$ & $-e$ & $-e$ & $-\1$ \\
        $\2$ & $\2\1$ & $e$ & $-\1$ & $\1$ & $\1\2$ & $-e$ & $-\2$ \\
        $\1\2$ & $-\2$ & $\1$ & $-e$ & $e$ & $\2$ & $-\1$ & $\2\1$ \\
        $\2\1$ & $\2$ & $-\1$ & $e$ & $-e$ & $-\2$ & $\1$ & $\1\2$ \\
        $-\1$ & $-e$ & $\2\1$ & $-\2$ & $\2$ & $e$ & $\1\2$ & $\1$ \\
        $-\2$ & $\1\2$ & $-e$ & $\1$ & $-\1$ & $\2\1$ & $e$ & $\2$ \\
        $-e$ & $-\1$ & $-\2$ & $\2\1$ & $\1\2$ & $\1$ & $\2$ & $e$ \\
    \end{tabular}

\end{document}

% vim: spell spelllang=en tw=79
