\documentclass[11pt, english, fleqn, DIV=15, headinclude, BCOR=1cm]{scrartcl}

\usepackage[
    bibatend,
    color,
]{../header}

\usepackage{booktabs}
\usepackage{pdflscape}

\usepackage{tikz}

\usepackage[tikz]{mdframed}
\newmdtheoremenv[%
    backgroundcolor=white,
    innertopmargin=\topskip,
    splittopskip=\topskip,
]{theorem}{Theorem}[section]

\newmdenv[%
    backgroundcolor=SframeBackground,
    frametitlebackgroundcolor=SframeTitleBackground,
    roundcorner=5pt,
    skipabove=\topskip,
    innertopmargin=\topskip,
    splittopskip=\topskip,
    frametitle={Problem statement},
    frametitlerule=true,
]{problem}

\newmdenv[%
    backgroundcolor=white,
    frametitlebackgroundcolor=SframeTitleBackground,
    roundcorner=5pt,
    skipabove=\topskip,
    innertopmargin=\topskip,
    innerbottommargin=8cm,
    splittopskip=\topskip,
    frametitle={Side question},
    frametitlerule=true,
    %nobreak=true,
]{question}


\hypersetup{
    pdftitle=
}

\newcommand\inv{^{-1}}

\newcounter{totalpoints}
\newcommand\punkte[1]{#1\addtocounter{totalpoints}{#1}}

\newcounter{problemset}
\setcounter{problemset}{7}

\subject{physics751 -- Group Theory}
\ihead{physics751 -- Problem Set \arabic{problemset}}

\title{Problem Set \arabic{problemset}}

\publishers{Group 1 -- Patrick Matuschek}
\ofoot{Group 1 -- Patrick Matuschek}

\author{
    Martin Ueding \\ \small{\href{mailto:mu@martin-ueding.de}{mu@martin-ueding.de}}
}
\ifoot{Martin Ueding}

\ohead{\rightmark}

\begin{document}

\maketitle

\section{The character table of $D_4$}

\subsection{Character table}

\begin{problem}
    Compute the character table for $D_4$.

    \emph{Hint}: Make use of the first orthogonality theorem for characters.
\end{problem}

The group $D_4$ is given by
\[
    D_4 = \{ e, c, c^2, c^3, b, bc, bc^2, bc^3 \}.
\]
The conjugacy classes are
\[
    c_1 = \{ e \}
    \eqnsep
    c_2 = \{ c, c^3 \}
    \eqnsep
    c_3 = \{ c^2 \}
    \eqnsep
    c_4 = \{ b, bc^2 \}
    \eqnsep
    c_5 = \{ bc, bc^3 \}.
\]

There are five classes, therefore the are five irreducible representations of
this group. Their dimensions must be
\[
    8 = 1 + 1 + 1 + 1 + 2^2.
\]

The dimensions alone give the $\chi_1$, since the class $c_1$ is the one that
contains the identity. The first represenation, $\Gamma_1$ is always the
trivial one, where everything is mapped to 1. Therefore, the characters for all
classes are 1. I summarize this information in the following character table:

\begin{tabular}{c|ccccc}
    & $\chi_1$ & $\chi_2$ & $\chi_3$ & $\chi_4$ & $\chi_5$ \\
    \midrule
    $\Gamma_1$ & 1 & 1 & 1 & 1 & 1 \\
    $\Gamma_2$ & 1 & & & & \\
    $\Gamma_3$ & 1 & & & & \\
    $\Gamma_4$ & 1 & & & & \\
    $\Gamma_5$ & 2 & & & & \\
\end{tabular}

In the one dimensional representations, where the group elements get mapped to
a scalar, the character is the representation element itself. Together with the
way the group product is mapped to the matrix multiplication, I get
\[
    \chi_2^{\Gamma_2}
    = \chi^{\Gamma_2}(c)
    = \chi^{\Gamma_2}(c^3)
    = \chi^{\Gamma_2}(c) \chi^{\Gamma_2}(c^2).
\]
This implies
\[
    \chi^{\Gamma_2}(c^2) = 1.
\]
That works in every one dimensional representation, so $\Gamma_1$ to
$\Gamma_4$. This gives more entries in the table. The orthonormality relation
for the columns gives me the $\chi_3^{\Gamma_5}$. This has to be $\pm 2$. Since
the columns have to be orthogonal, only $-2$ is a viable option.

\begin{tabular}{c|SSSSS}
    & {$\chi_1$} & {$\chi_2$} & {$\chi_3$} & {$\chi_4$} & {$\chi_5$} \\
    \midrule
    $\Gamma_1$ & 1 & 1 & 1 & 1 & 1 \\
    $\Gamma_2$ & 1 & & 1 & & \\
    $\Gamma_3$ & 1 & & 1 & & \\
    $\Gamma_4$ & 1 & & 1 & & \\
    $\Gamma_5$ & 2 & & -2 & & \\
\end{tabular}

The row sum of squared characters (times number of elements on the class) in
the last row is already 8. That means that all other characters have to be
zero.

$b$ is its own inverse. Its character in the one dimensional representations
can only be $\pm 1$ then. For the classes $c_2$ and $c_5$, the order of the
elements is 4. Since I assume a real carrier space (can I always do that?), the
only options are $\pm 1$ as well.

The table looks like this now:

\begin{tabular}{c|SSSSS}
    & {$\chi_1$} & {$\chi_2$} & {$\chi_3$} & {$\chi_4$} & {$\chi_5$} \\
    \midrule
    $\Gamma_1$ & 1 & 1 & 1 & 1 & 1 \\
    $\Gamma_2$ & 1 & \pm 1 & 1 & \pm 1 & \pm 1 \\
    $\Gamma_3$ & 1 & \pm 1 & 1 & \pm 1 & \pm 1 \\
    $\Gamma_4$ & 1 & \pm 1 & 1 & \pm 1 & \pm 1 \\
    $\Gamma_5$ & 2 & 0 & -2 & 0 & 0 \\
\end{tabular}

The squared row and column sums all check out to be 8, so the magnitudes of all
characters seems to be fine. Now the signs have to be chosen in a way that all
the rows and columns are orthonormal. I chose the signs stepwise such that the
new row is orthogonal to the ones that are fixed until then. This is the table
that I got:

\begin{tabular}{c|SSSSS}
    & {$\chi_1$} & {$\chi_2$} & {$\chi_3$} & {$\chi_4$} & {$\chi_5$} \\
    \midrule
    $\Gamma_1$ & 1 & 1 & 1 & 1 & 1 \\
    $\Gamma_2$ & 1 & 1 & 1 & -1 & -1 \\
    $\Gamma_3$ & 1 & -1 & 1 & 1 & -1 \\
    $\Gamma_4$ & 1 & -1 & 1 & -1 & 1 \\
    $\Gamma_5$ & 2 & 0 & -2 & 0 & 0 \\
\end{tabular}

As far as I can tell, all the rows and columns are orthogonal and sum to 8 in
the appropriate manner.

\begin{question}
    Maybe there is a more systematic way to chose the signs. I set up a system
    of equations for all the scalar products that have to vanish. Since this
    was a non linear system of equations I did not know of any systematic
    approach. Is there one?
\end{question}

\subsection{Representation product}

\begin{problem}
    Let $D^{(5)}$ denote the two-dimensional irreducible representation of
    $D_4$. Check whether the product $D^{(5)} \otimes D^{(5)}$ is reducible and
    calculate the Clebsch-Gordan decomposition if necessary.
\end{problem}

The expansion works like this:
\[
    D^{\Gamma_5 \times \Gamma_5}
    = D^{\Gamma_5} \otimes D^{\Gamma_5}
    = \sum_i a_i D^{\Gamma_i} 
\]

The expansion coefficients can be computed with
\[
    a_i = \bracket{\vec\chi^{\Gamma_i}, \vec\chi^{\Gamma_5} \vec\chi^{\Gamma_5}}.
\]

The character vectors are given by:
\begin{gather*}
    \vec\chi^{\Gamma_1} = \begin{pmatrix} 1 & 1 & 1 & 1 & 1 \end{pmatrix} \\
    \vec\chi^{\Gamma_2} = \begin{pmatrix} 1 & 1 & 1 & -1 & -1 \end{pmatrix} \\
    \vec\chi^{\Gamma_3} = \begin{pmatrix} 1 & -1 & 1 & 1 & -1 \end{pmatrix} \\
    \vec\chi^{\Gamma_4} = \begin{pmatrix} 1 & -1 & 1 & -1 & 1 \end{pmatrix} \\
    \vec\chi^{\Gamma_5} = \begin{pmatrix} 2 & 0 & -2 & 0 & 0 \end{pmatrix}
\end{gather*}
The order of the group has to be divided out, the number of elements on each
class has to be taken into account.

All that is left is to compute the scalar products. The second term is
\[
    \vec\chi^{\Gamma_5} \vec\chi^{\Gamma_5} =
    \begin{pmatrix}
        4 & 0 & 4 & 0 & 0
    \end{pmatrix}.
\]
Now I can compute the coefficients.
\begin{align*}
    a_1
    &= \bracket{\vec\chi^{\Gamma_1}, \vec\chi^{\Gamma_5} \vec\chi^{\Gamma_5}} \\
    &= \frac{1}{8}[1 \cdot 1 \cdot 4 + 1 \cdot 1 \cdot 4] \\
    &= 1 \\
    \intertext{%
        Since the character vectors do not differ from the first one, the
        scalar products are all the same.
    }
    a_2 &= 1 \\
    a_3 &= 1 \\
    a_4 &= 1 \\
    a_5
    &= \bracket{\vec\chi^{\Gamma_1}, \vec\chi^{\Gamma_5} \vec\chi^{\Gamma_5}} \\
    &= \frac{1}{8}[1 \cdot 2 \cdot 4 + 1 \cdot [-2] \cdot 4] \\
    &= 0
\end{align*}
So I end up with
\[
    D^{\Gamma_5 \times \Gamma_5}
    = D^{\Gamma_5} \otimes D^{\Gamma_5}
    = D^{\Gamma_1} + D^{\Gamma_2} + D^{\Gamma_3} + D^{\Gamma_4}.
\]

\end{document}

% vim: spell spelllang=en tw=79
