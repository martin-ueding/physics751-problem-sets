\documentclass[11pt, english, fleqn, DIV=15, headinclude, BCOR=1cm]{scrartcl}

\usepackage[
    bibatend,
    color,
]{../header}

\usepackage{booktabs}
\usepackage{pdflscape}

\usepackage{tikz}

\usepackage[tikz]{mdframed}
\newmdtheoremenv[%
    backgroundcolor=white,
    innertopmargin=\topskip,
    splittopskip=\topskip,
]{theorem}{Theorem}[section]

\newmdenv[%
    backgroundcolor=SframeBackground,
    frametitlebackgroundcolor=SframeTitleBackground,
    roundcorner=5pt,
    skipabove=\topskip,
    innertopmargin=\topskip,
    splittopskip=\topskip,
    frametitle={Problem statement},
    frametitlerule=true,
]{problem}

\newmdenv[%
    backgroundcolor=white,
    frametitlebackgroundcolor=SframeTitleBackground,
    roundcorner=5pt,
    skipabove=\topskip,
    innertopmargin=\topskip,
    innerbottommargin=8cm,
    splittopskip=\topskip,
    frametitle={Side question},
    frametitlerule=true,
    %nobreak=true,
]{question}

\DeclareMathOperator{\Tr}{Tr}


\hypersetup{
    pdftitle=
}

\newcommand\inv{^{-1}}

\newcounter{totalpoints}
\newcommand\punkte[1]{#1\addtocounter{totalpoints}{#1}}

\newcounter{problemset}
\setcounter{problemset}{9}

\subject{physics751 -- Group Theory}
\ihead{physics751 -- Problem Set \arabic{problemset}}

\title{Problem Set \arabic{problemset}}

\publishers{Group 1 -- Patrick Matuschek}
\ofoot{Group 1 -- Patrick Matuschek}

\author{
    Martin Ueding \\ \small{\href{mailto:mu@martin-ueding.de}{mu@martin-ueding.de}}
}
\ifoot{Martin Ueding}

\ohead{\rightmark}

\begin{document}

\maketitle

\section{Charge operator}

\begin{problem}
    Determine the expectation values of $\hat q(3)$.
\end{problem}

The correct normalization that should have been calculated in the presence
problems is given by a factor $1/\sqrt 2$.

\newcommand\Pp[1]{\Psi_\text{proton}^{[M,#1]}}

\begin{align*}
    \Braket{\Pp S | \hat q(3) | \Pp S}
    &= \frac 12 \Braket{\text{udu} - \text{duu} | \hat q(3) | \text{udu} - \text{duu}} \\
    \intertext{%
        I use the linearity of the scalar product. I further assume that any
        flavor states that are not exactly the same are orthogonal. Therefore,
        only those elements remain:
    }
    &= \frac 12 \Braket{\text{udu} | \hat q(3) | \text{udu}}
    + \frac 12 \Braket{\text{duu} | \hat q(3) | \text{duu}}. \\
    \intertext{%
        Now I apply the charge operator and obtain
    }
    &= \frac 12 \Braket{\text{udu} | \frac 23 | \text{udu}}
    + \frac 12 \Braket{\text{duu} | \frac 23 | \text{duu}}. \\
    \intertext{%
        I pull the fraction out front.
    }
    &= \frac 13 \Braket{\text{udu} | \text{udu}}
    + \frac 13 \Braket{\text{duu} | \text{duu}} \\
    \intertext{%
        Those scalar products are just unity, so only
    }
    &= \frac 23
\end{align*}
remains.

For the second state, the normalization is $1/\sqrt 6$. The matrix element is:
\begin{align*}
    \Braket{\Pp A | \hat q(3) | \Pp A}
    &= \frac{1}{6} \braket{2 \text{uud} - \text{udu} - \text{duu} | \hat q(3) | 2
    \text{uud} - \text{udu} - \text{duu}} \\
    \intertext{%
        I use the same orthogonality and linearity here as well.
    }
    &= \frac{1}{6} \braket{2 \text{uud} | \hat q(3) | 2 \text{uud}}
    + \frac{1}{6} \braket{\text{udu} | \hat q(3) | \text{udu}}
    + \frac{1}{6} \braket{\text{duu} | \hat q(3) | \text{duu}} \\
    \intertext{%
        I now apply the charge operator to the last quark.
    }
    &= \frac{1}{6} \Braket{2 \text{uud} | -\frac13 | 2 \text{uud}}
    + \frac{1}{6} \Braket{\text{udu} | \frac 23 | \text{udu}}
    + \frac{1}{6} \Braket{\text{duu} | \frac 23 | \text{duu}} \\
    \intertext{%
        The first scalar product is 4, the last two are 1. This leaves
    }
    &= \frac 16 \sbr{- \frac 43 + \frac 23 + \frac 23} \\
    &= 0.
\end{align*}

\end{document}

% vim: spell spelllang=en tw=79
