\documentclass[11pt, english, fleqn, DIV=15, headinclude, BCOR=1cm]{scrartcl}

\usepackage[
    bibatend,
    %color,
]{../header}

\usepackage{booktabs}
\usepackage{pdflscape}

\usepackage{tikz}

\usepackage[tikz]{mdframed}
\newmdtheoremenv[%
    backgroundcolor=white,
    innertopmargin=\topskip,
    splittopskip=\topskip,
]{theorem}{Theorem}[section]

\newmdenv[%
    backgroundcolor=SframeBackground,
    frametitlebackgroundcolor=SframeTitleBackground,
    roundcorner=5pt,
    skipabove=\topskip,
    innertopmargin=\topskip,
    splittopskip=\topskip,
    frametitle={Problem statement},
    frametitlerule=true,
]{problem}

\newmdenv[%
    backgroundcolor=white,
    frametitlebackgroundcolor=SframeTitleBackground,
    roundcorner=5pt,
    skipabove=\topskip,
    innertopmargin=\topskip,
    innerbottommargin=8cm,
    splittopskip=\topskip,
    frametitle={Side question},
    frametitlerule=true,
    %nobreak=true,
]{question}

\DeclareMathOperator{\Tr}{Tr}


\hypersetup{
    pdftitle=
}

\newcommand\inv{^{-1}}

\newcounter{totalpoints}
\newcommand\punkte[1]{#1\addtocounter{totalpoints}{#1}}

\newcounter{problemset}
\setcounter{problemset}{12}

\subject{physics751 -- Group Theory}
\ihead{physics751 -- Problem Set \arabic{problemset}}

\title{Problem Set \arabic{problemset}}

\publishers{Group 1 -- Patrick Matuschek}
\ofoot{Group 1 -- Patrick Matuschek}

\author{
    Martin Ueding \\ \small{\href{mailto:mu@martin-ueding.de}{mu@martin-ueding.de}}
}
\ifoot{Martin Ueding}

\ohead{\rightmark}

\usepackage[vcentermath]{youngtab}

\begin{document}

\maketitle

\section{Tensor products of Young tableaux}

\subsection{}

\begin{problem}
    Calculate $\bar 3 \otimes \bar 3$ and $\bar 3 \otimes \bar 3 \otimes \bar
    3$.
\end{problem}

To solve arbitrary tensor products I distilled the algorithm given on the
problem set into an interactive website which can be found at:

\url{http://app.martin-ueding.de/young-frame-product/}

Or just scan the following QR code:

\includegraphics[height=3cm]{app.png}

\subsubsection{Calculation of $\bar 3 \otimes \bar 3$}

Using my program, I get the following:

\[
    \yng(1,1) \otimes \yng(1,1)
    = \yng(2,2) \oplus \yng(2,1,1)
    = \yng(2,2) \oplus \yng(1).
\]
In the last step, it is used that $N = 3$ and the first column can therefore be
omitted. This can also be rewritten in terms of the dimensions:
\[
    \bar 3 \otimes \bar 3 = 6 \oplus 3.
\]

\subsubsection{Calculation of $\bar 3 \otimes \bar 3 \otimes \bar 3$}

Now this has to be multiplied with $\bar 3$ again. First, I expand this:
\begin{align*}
    \sbr{\yng(2,2) \oplus \yng(1)} \otimes \yng(1,1)
    &= \yng(2,2) \otimes \yng(1,1) \oplus \yng(1) \otimes \yng(1,1)
    \intertext{%
        Now I use my program to compute the products of this. I will put
        brackets around the summands of the above equation make it clear which
        belongs to which.
    }
    &= \sbr{ \yng(3,3) \oplus \yng(3,2,1) } \oplus \sbr{ \yng(2,1) \oplus
    \yng(1,1,1) }
    \intertext{%
        The second summand can be simplified and I am left with:
    }
    &= \yng(3,3) \oplus 2 \, \yng(2,1) \oplus \yng(1,1,1)
    \intertext{%
        The program also gives the dimensions, so I can write this as:
    }
    &= 10 \oplus 8^2 \oplus 1
\end{align*}
The total dimension is 27, so this checks out.

\subsection{}

\begin{problem}
    Calculate $8 \otimes 8$.
\end{problem}

The expected dimension would be 64. This is the output of the program:
\[
    \yng(2,1) \otimes \yng(2,1)
    =
    \yng(4,2) \oplus \yng(4,1,1) \oplus \yng(3,3) \oplus 2 \, \yng(3,2,1)
    \oplus \yng(2,2,2)
\]
The program then computed the dimensions and gets:
\[
    27 \oplus 10 \oplus 10 \oplus 8^2 \oplus 1
\]
This sum is 64, so it matches the dimensions.

\subsection{Exact SU(3) symmetry}

In the presence problem A~13.3 we looked at the scattering of two open charm
mesons with each other. Because we assumed the exact SU(3) symmetry and there
are three of those D quarks, we need a three dimensional representation. The
one that applied here is the $\Gamma_3$ one. Then we took $3 \otimes 3 = 6
\oplus \bar 3$ and said that there are two irreducible representations,
therefore there are two independent scattering amplitudes.

\begin{question}
    Did I say this correctly?
\end{question}

Now we scatter one of those D quarks with anti D quarks. There is another three
dimensional representation, the $\Gamma_{\bar 3}$. With this, I can do the same
calculation:
\[
    3 \otimes \bar 3 = 8 \oplus 1
    \eqnsep
    \yng(1) \otimes \yng(1,1) = \yng(2,1) \oplus \yng(1,1,1)
\]
So there are two irreducible representations here as well. Therefore, there are
two scattering amplitudes as well.

\end{document}

% vim: spell spelllang=en tw=79
