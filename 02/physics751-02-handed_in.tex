\documentclass[11pt, ngerman, fleqn, DIV=15, headinclude, BCOR=1cm]{scrartcl}

\usepackage[bibatend]{../header}

\usepackage{booktabs}

\usepackage{tikz}

\usepackage[tikz]{mdframed}
\newmdtheoremenv[%
    backgroundcolor=black!5,
    innertopmargin=\topskip,
    splittopskip=\topskip,
]{theorem}{Theorem}[section]

\newmdenv[%
    backgroundcolor=black!5,
    frametitlebackgroundcolor=black!10,
    roundcorner=5pt,
    innertopmargin=\topskip,
    splittopskip=\topskip,
    frametitle={Problem Statement},
    frametitlerule=true,
]{problem}

\hypersetup{
    pdftitle=
}

\newcommand\inv{^{-1}}

\newcounter{totalpoints}
\newcommand\punkte[1]{#1\addtocounter{totalpoints}{#1}}

\newcounter{problemset}
\setcounter{problemset}{2}

\subject{physics751 -- Group Theory}
\ihead{physics751 -- Problem Set \arabic{problemset}}

\title{Problem Set \arabic{problemset}}

\publishers{Group 1 -- Patrick Matuschek}
\ofoot{Group 1 -- Patrick Matuschek}



\author{
    Martin Ueding \\ \small{\href{mailto:mu@martin-ueding.de}{mu@martin-ueding.de}}
}
\ifoot{Martin Ueding}

\ohead{\rightmark}

\begin{document}

\maketitle

\vspace{3ex}

\begin{center}
    \begin{tabular}{rrr}
        problem number & achieved points & possible points \\
        \midrule
        H \arabic{problemset}.1 & &  \\
        \midrule
        total & & 
    \end{tabular}
\end{center}

\section{The Alternating Group $\mathcal A_N$}

\subsection{Subgroup}

\begin{problem}
    Prove that the set of all even permutations forms a subgroup of the
    symmetric group $\mathcal S_N$.
\end{problem}

As always, there are the four group axioms to show.

\begin{description}
    \item[Closure]
        Given a permutation $P$, it can be broken down into cycles. The number
        of elements in the $i$th cycle is $n_i$. So $(123)$ has $n = 3$. The
        order of the whole permutation was given in the lecture as
        \[
            W = \sum_i [n_1 - 1].
        \]

        When another permutation is added, the cycles are written as a
        juxtaposition. The order will be the sum of the respective orders of
        the permutations. The sum of two even numbers is again even, so the
        resulting permutation is another even permutation.

    \item[Unity]
        The cycle $(1)$ is even and the neutral element.

    \item[Inverse]
        For any cycle, the inverse can be formed with elements reversed. The
        inverse of multiple cycle is the reverse order of inverse cycles.
        The order of the permutation does not change in this way. Therefore, an
        inverse can always be found among the even permutations.

    \item[Associativity]
        This subgroup in spe takes it juxtaposition operation $\circ$ from the
        symmetric group which was associative with any permutations. This
        property is not list when only a subset of the permutations is taken
        into account.
\end{description}

Since all axioms are fulfilled, this subset forms a group itself. I think that
it even is a normal subgroup, since $\bar P \mathcal S_N = \mathcal S_N \bar
P$, which is also somewhat used in the next part of this problem.

\subsection{Order}

\begin{problem}
    Derive its order.
\end{problem}

Any even permutations directly turns into an odd one when the cycle $(12)$ is
applied to it. Any odd permutation turns into an even one when the same cycle
is applied. Applying it twice gives unity: $(12)^2 = e$.

I think that $(12) \circ \mathcal A_N$ will give a set of only odd
permutations. Those do not form a group because the closure cannot be met. The
combination of two odd permutation gives an even permutation. However, when
$(12)$ is applied to that set again, it must be $\mathcal A_N$ again, since
only the unity $e$ was applied to it. By association law,
\[
    [(12) (12)] \mathcal A_N = (12) (12) \mathcal A_N,
\]
where the multiplication is read in the usual right-to-left way on the right
side. Since the order of $\mathcal A_N$ must not change when applying $e$ to it
(rearrangement theorem), the order of $(12) \mathcal A_N$ must the same as
$\mathcal A_N$ itself. That would be $N!/2$.

\end{document}

% vim: spell spelllang=en tw=79
