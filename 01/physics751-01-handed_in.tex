\documentclass[11pt, ngerman, fleqn, DIV=15, headinclude, BCOR=1cm]{scrartcl}

\usepackage[bibatend]{../header}

\usepackage{booktabs}

\usepackage{tikz}

\usepackage[tikz]{mdframed}
\newmdtheoremenv[%
    backgroundcolor=black!5,
    innertopmargin=\topskip,
    splittopskip=\topskip,
]{theorem}{Theorem}[section]

\newmdenv[%
    backgroundcolor=black!5,
    frametitlebackgroundcolor=black!10,
    roundcorner=5pt,
    innertopmargin=\topskip,
    splittopskip=\topskip,
    frametitle={Problem Statement},
    frametitlerule=true,
]{problem}

\hypersetup{
    pdftitle=
}

\newcommand\inv{^{-1}}

\newcounter{totalpoints}
\newcommand\punkte[1]{#1\addtocounter{totalpoints}{#1}}

\newcounter{problemset}
\setcounter{problemset}{1}

\subject{physics751 -- Group Theory}
\ihead{physics751 -- Problem Set \arabic{problemset}}

\title{Problem Set \arabic{problemset}}

\publishers{Group 1 -- Patrick Matuschek}
\ofoot{Group 1 -- Patrick Matuschek}



\author{
    Martin Ueding \\ \small{\href{mailto:mu@martin-ueding.de}{mu@martin-ueding.de}}
}
\ifoot{Martin Ueding}

\ohead{\rightmark}

\begin{document}

\maketitle

\vspace{3ex}

\begin{center}
    \begin{tabular}{rrr}
        problem number & achieved points & possible points \\
        \midrule
        H 1.1 & & \punkte{0} \\
        \midrule
        total & & \arabic{totalpoints}
    \end{tabular}
\end{center}

\vspace{5ex}

I, Martin Ueding, would like to scan and upload the problem sets with your
corrections to my website \href{http://martin-ueding.de}{martin-ueding.de}.
There, the original problem set as well as the reviewed one will be licensed
under the “\href{http://creativecommons.org/licenses/by-sa/4.0/}{Creative
Commons Attribution-ShareAlike 4.0 International License}”. Is that okay with
you?

Yes $\Box$ \hspace{2cm} No $\Box$

\newpage

\section{Vierergruppe (Klein Four-Group)}

\subsection{Abelian}

\begin{problem}
    Prove that all groups with at most four elements are abelian. \\
    Hint: Study the groups with three elements $\{e, a, b\}$ first.
\end{problem}

I start with a group consisting of three elements, $\{e, a, b\}$.
\begin{itemize}
    \item
        The elements are unique, as the above is a set.
    \item
        There must be an inverse of $a$ to make it a group.
    \item
        $a\inv$ must be $b$, since $e$ cannot be the inverse. If that would be
        the case, then $ea=e$ and not $ea=a$ as expected from the identity
        element.
    \item
        The left inverse, $b$, must also be a right inverse.
    \item
        Therefore, $a$ commutes with $b$ since they are inverses of each other.
\end{itemize}

Now I will do this with four elements: $\{e, a, b, c\}$. Let the inverse of $a$
be $b$ again. $c$ has to have an inverse itself to satisfy that group axiom.
There are three possibilities:

\begin{description}
    \item[Case $c\inv = a$]
        \begin{align*}
            a &= c\inv. \\
            \intertext{%
                Premuliply with $a\inv$:
            }
            a\inv a &= a\inv c\inv. \\
            \intertext{%
                Postmuliply with $c$:
            }
            a\inv a c &= a\inv c\inv c. \\
            \intertext{%
                Left side: Using the fact that $c$ is the inverse of $a$. Right
                side: Same thing, just that $a\inv c\inv = (ca)\inv = e\inv =
                e$. I get
            }
            a\inv &= c.
        \end{align*}
        The inverse for $a$ is now $c$. By previous choice, the inverse of $a$
        was $b$. This means that $b = c$, which is not allowed since elements
        of a set must be unique.

    \item[Case $c\inv = b$]
        Same as above, just that $a = b$.

    \item[Case $c\inv = c$]
        This is the only possibility left.
\end{description}

As before, $a$ commutes with $b$. The commutation with $c$ is not clear yet.
First I have to define what $ac$ and $bc$ is supposed to be.

\subsection{Isomorphic}

\begin{problem}
    Show that every group with four elements is isomorphic to either $\Z_4$
    (cyclic group of four elements or to $\Z_2 \times \Z_2$ (Klein Four-Group).
    Give the multiplication tables.
\end{problem}

One possible representation of $\Z_4$ is $\{1, \iup, -1, -\iup\}$ with the
complex product as association $\circ$. I will call the elements $\{e, a, b,
c\}$ and write down the multiplication matrix:
\[
    \tens T_{\Z_4} =
    \begin{pmatrix}
        e & a & b & c \\
        a & b & c & e \\
        b & c & e & a \\
        c & e & a & b
    \end{pmatrix}
\]

\subsection{Comparison to other groups}

\begin{problem}
    Compare the groups found in point 2 with the groups acting on directed and
    undirected polyhedra.
\end{problem}

\end{document}

% vim: spell spelllang=en tw=79
