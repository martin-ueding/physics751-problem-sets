\documentclass[11pt, english, fleqn, DIV=15, headinclude, BCOR=1cm]{scrartcl}

\usepackage[
    bibatend,
    color,
]{../header}

\usepackage{booktabs}
\usepackage{pdflscape}

\usepackage{tikz}

\usepackage[tikz]{mdframed}
\newmdtheoremenv[%
    backgroundcolor=black!5,
    innertopmargin=\topskip,
    splittopskip=\topskip,
]{theorem}{Theorem}[section]

\newmdenv[%
    backgroundcolor=black!5,
    frametitlebackgroundcolor=black!10,
    roundcorner=5pt,
    skipabove=\topskip,
    innertopmargin=\topskip,
    splittopskip=\topskip,
    frametitle={Problem statement},
    frametitlerule=true,
]{problem}

\newmdenv[%
    backgroundcolor=white,
    frametitlebackgroundcolor=black!10,
    roundcorner=5pt,
    skipabove=\topskip,
    innertopmargin=\topskip,
    innerbottommargin=8cm,
    splittopskip=\topskip,
    frametitle={Side question},
    frametitlerule=true,
    %nobreak=true,
]{question}


\hypersetup{
    pdftitle=
}

\newcommand\inv{^{-1}}

\newcounter{totalpoints}
\newcommand\punkte[1]{#1\addtocounter{totalpoints}{#1}}

\newcounter{problemset}
\setcounter{problemset}{4}

\subject{physics751 -- Group Theory}
\ihead{physics751 -- Problem Set \arabic{problemset}}

\title{Problem Set \arabic{problemset}}

\publishers{Group 1 -- Patrick Matuschek}
\ofoot{Group 1 -- Patrick Matuschek}

\author{
    Martin Ueding \\ \small{\href{mailto:mu@martin-ueding.de}{mu@martin-ueding.de}}
}
\ifoot{Martin Ueding}

\ohead{\rightmark}

\begin{document}

\maketitle

\begin{question}
    During the lecture the representation of the symmetric group $\mathcal S_2$
    was given. The multiplication table was written down:
    \[
        T_{\mathcal S_2} =
        \begin{pmatrix}
            e & (12) \\
            (12) & e
        \end{pmatrix}.
    \]

    From there, the representation was given as
    \[
        D^{[]}(e) =
        \begin{pmatrix}
            1 & 0 \\ 0 & 1
        \end{pmatrix}
        \eqnsep
        D^{[]}((12)) =
        \begin{pmatrix}
            0 & 1 \\ 1 & 0
        \end{pmatrix}.
    \]

    The definition given in the lecture was, as far as I recall,
    \[
        D_{jk}^{[]} (g_i) =
        \begin{cases}
            1 & g_j g_k = g_i \\
            0 & \text{else}
        \end{cases}.
    \]
    In words, this would mean that the representation matrices for a group
    element is a matrix that just had ones everywhere where that group element
    appears in the multiplication table, zeros everywhere else. This is fine
    with me.

    Then the lecturer said that on some occasions the multiplication table has
    to be rearranged in a way such that the identity is on the diagonal such
    that $D(e) = \mathbb 1$. The rearrangement theorem asserts that this
    rearrangement of rows and columns is possible. What concerns me is the fact
    that this rearranging might make the indexing of group elements
    ill-defined.

    The definition of multiplication table of a group $G$ with elements $g_i$
    that I remember is $T_{ij} := g_i g_j$. For this, there has so be something
    that maps the \emph{set} of group elements into an \emph{ordered tuple}
    such that $g_i$ is always the same element. Since the order itself is
    arbitrary, I do not mind when a permutation $\pi \in \mathcal S_{|G|}$ is
    applied such that
    \[
        G = \{ g_i\colon i \leq |G| \} = \{ g_{\pi(i)} \colon i \leq |G| \}
    \]
    still holds.

    We were all asked to derive the representation of $\mathcal S_3$ at home,
    so that is what I attempted.

    This is my $\mathcal S_3$ and the order in the set is the order I use for
    my indexing with $i$:
    \[
        \{ (1), (12), (13), (23), (123), (321) \}.
    \]
    So those are $g_1$ to $g_6$, in that order. Now I computed the
    multiplication table using the fast multiplication and the rearrangement
    theorem. This is the table that I got:

    \begin{tabular}{c|cccccc}
        $     $ & $(1  )$ & $(12 )$ & $(13 )$ & $(23 )$ & $(123)$ & $(321)$ \\
        \midrule
        $(1  )$ & $(1  )$ & $(12 )$ & $(13 )$ & $(23 )$ & $(123)$ & $(321)$ \\
        $(12 )$ & $(12 )$ & $(1  )$ & $(321)$ & $(123)$ & $(32 )$ & $(13 )$ \\
        $(13 )$ & $(13 )$ & $(123)$ & $(1  )$ & $(321)$ & $(12 )$ & $(23 )$ \\
        $(23 )$ & $(23 )$ & $(321)$ & $(123)$ & $(1  )$ & $(13 )$ & $(12 )$ \\
        $(123)$ & $(123)$ & $(13 )$ & $(23 )$ & $(12 )$ & $(321)$ & $(1  )$ \\
        $(321)$ & $(321)$ & $(23 )$ & $(12 )$ & $(13 )$ & $(1  )$ & $(123)$ \\
    \end{tabular}

    To get the unity elements on the diagonal only, I had to exchange the last
    two rows. One could also exchange the last two columns, but not both. This
    is the changed table.

    \begin{tabular}{c|cccccc}
        $     $ & $(1  )$ & $(12 )$ & $(13 )$ & $(23 )$ & \textbf{(123)} & \textbf{(321)} \\
        \midrule
        $(1  )$ & $(1  )$ & $(12 )$ & $(13 )$ & $(23 )$ & $(123)$ & $(321)$ \\
        $(12 )$ & $(12 )$ & $(1  )$ & $(321)$ & $(123)$ & $(32 )$ & $(13 )$ \\
        $(13 )$ & $(13 )$ & $(123)$ & $(1  )$ & $(321)$ & $(12 )$ & $(23 )$ \\
        $(23 )$ & $(23 )$ & $(321)$ & $(123)$ & $(1  )$ & $(13 )$ & $(12 )$ \\
        \textbf{(321)} & $(321)$ & $(23 )$ & $(12 )$ & $(13 )$ & $(1  )$ & $(123)$ \\
        \textbf{(123)} & $(123)$ & $(13 )$ & $(23 )$ & $(12 )$ & $(321)$ & $(1  )$ \\
    \end{tabular}

    At this point, I do not know that the meaning of $g_5$ and $g_6$ are
    supposed to be now.

    Then I tried the more formal way and took the definition. Using the
    ordering that I defined above, $D(e)$ will not be $\mathbb 1$. The only way
    I see to get it to work with the definition is to apply a permutation such
    that the definition is not
    \[
        D_{jk}^{[]} (g_i) =
        \begin{cases}
            1 & g_{\pi(j)} g_{\pi(k)} = g_{\pi(i)} \\
            0 & \text{else}
        \end{cases}
    \]
    but something like
    \[
        D_{jk}^{[]} (g_i) =
        \begin{cases}
            1 & g_{\pi(j)} g_{k} = g_{i} \\
            0 & \text{else}
        \end{cases}
    \]
    where only one element is permuted. To me, this looks like $s_j$ does not
    need to be $s_k$ even if $j = k$.

    What is the correct way of doing this?
\end{question}

\section{Pauli matrices and the quaternion group $Q_4$}

\end{document}

% vim: spell spelllang=en tw=79
