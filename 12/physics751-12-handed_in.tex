\documentclass[11pt, english, fleqn, DIV=15, headinclude, BCOR=1cm]{scrartcl}

\usepackage[
    bibatend,
    color,
]{../header}

\usepackage{booktabs}
\usepackage{pdflscape}

\usepackage{tikz}

\usepackage[tikz]{mdframed}
\newmdtheoremenv[%
    backgroundcolor=white,
    innertopmargin=\topskip,
    splittopskip=\topskip,
]{theorem}{Theorem}[section]

\newmdenv[%
    backgroundcolor=SframeBackground,
    frametitlebackgroundcolor=SframeTitleBackground,
    roundcorner=5pt,
    skipabove=\topskip,
    innertopmargin=\topskip,
    splittopskip=\topskip,
    frametitle={Problem statement},
    frametitlerule=true,
]{problem}

\newmdenv[%
    backgroundcolor=white,
    frametitlebackgroundcolor=SframeTitleBackground,
    roundcorner=5pt,
    skipabove=\topskip,
    innertopmargin=\topskip,
    innerbottommargin=8cm,
    splittopskip=\topskip,
    frametitle={Side question},
    frametitlerule=true,
    %nobreak=true,
]{question}

\DeclareMathOperator{\Tr}{Tr}


\hypersetup{
    pdftitle=
}

\newcommand\inv{^{-1}}

\newcounter{totalpoints}
\newcommand\punkte[1]{#1\addtocounter{totalpoints}{#1}}

\newcounter{problemset}
\setcounter{problemset}{12}

\subject{physics751 -- Group Theory}
\ihead{physics751 -- Problem Set \arabic{problemset}}

\title{Problem Set \arabic{problemset}}

\publishers{Group 1 -- Patrick Matuschek}
\ofoot{Group 1 -- Patrick Matuschek}

\author{
    Martin Ueding \\ \small{\href{mailto:mu@martin-ueding.de}{mu@martin-ueding.de}}
}
\ifoot{Martin Ueding}

\ohead{\rightmark}

\begin{document}

\maketitle

\section{Momentum operator}

I copy the gradient from \parencite{wikipedia/del_spherical}:
\[
    \vnabla =
    \ev_r \pd{}r
    + \ev_\theta \frac 1r \pd{}\theta
    + \ev_\phi \frac1{r \sin(\theta)} \pd{}\phi.
\]
Then with $\vec r = r \ev_r$ I can form the cross product, which is linear in
its two arguments:
\begin{align*}
    \vec r \times \vnabla
    &= r \ev_r \times \sbr{\ev_r \pd{}r + \ev_\theta \frac 1r \pd{}\theta
    + \ev_\phi \frac1{r \sin(\theta)} \pd{}\phi}. \\
    &= r \ev_r \times \ev_\theta \frac 1r \pd{}\theta
    + r \ev_r \times \ev_\phi \frac1{r \sin(\theta)} \pd{}\phi
    \intertext{%
        Cancel $r$.
    }
    &= \ev_r \times \ev_\theta \pd{}\theta
    + \ev_r \times \ev_\phi \frac1{\sin(\theta)} \pd{}\phi
    \intertext{%
        The vectors $\{ \ev_r, \ev_\theta, \ev_\phi \}$ form a right-handed
        orthonormal basis. That means that those cross products are simply
        cyclic.
    }
    &= \ev_\phi \pd{}\theta - \ev_\theta \frac1{\sin(\theta)} \pd{}\phi
\end{align*}

\begin{landscape}

\section{Laplacian in three dimensions}

\begin{align*}
    \vec L^2
    &= - \sbr{\ev_\phi \pd{}\theta - \ev_\theta \frac1{\sin(\theta)}
    \pd{}\phi}^2 \\
    &= - \sbr{\ev_\phi \pd{}\theta - \ev_\theta \frac1{\sin(\theta)}
    \pd{}\phi} \sbr{\ev_\phi \pd{}\theta - \ev_\theta \frac1{\sin(\theta)}
    \pd{}\phi} \\
    &= -
    \sbr{
        \ev_\phi \pd{}\theta
        \ev_\phi \pd{}\theta
        -
        \ev_\phi \pd{}\theta
        \ev_\theta \frac1{\sin(\theta)} \pd{}\phi
        -
        \ev_\theta \frac1{\sin(\theta)} \pd{}\phi
        \ev_\phi \pd{}\theta
        +
        \ev_\theta \frac1{\sin(\theta)} \pd{}\phi
        \ev_\theta \frac1{\sin(\theta)} \pd{}\phi
    }
    \intertext{%
        Now I have apply the product rule to get even more terms. I will not do
        them all at once since the page is not wide enough, even in landscape.
        The first two look like this when expanded:
    }
    &= -
    \sbr{
        \ev_\phi \sbr{\pd{}\theta \ev_\phi} \pd{}\theta
        +
        \ev_\phi \ev_\phi \pd{^2}{\theta^2}
        -
        \ev_\phi \ev_\theta \pd{}\theta \frac1{\sin(\theta)} \pd{}\phi
        -
        \ev_\phi \sbr{\pd{}\theta \ev_\theta} \frac1{\sin(\theta)} \pd{}\phi
        -
        \ev_\theta \frac1{\sin(\theta)} \pd{}\phi
        \ev_\phi \pd{}\theta
        +
        \ev_\theta \frac1{\sin(\theta)} \pd{}\phi
        \ev_\theta \frac1{\sin(\theta)} \pd{}\phi
    }
    \intertext{%
        The first term is zero since $\ev_\phi$ has no dependence on $\theta$.
        The second one can be simplified since the vectors are unit vectors.
        The third term is zero because of the orthogonality of the basis
        vectors.
    }
    &= -
    \sbr{
        \pd{^2}{\theta^2}
        -
        \ev_\phi \sbr{\pd{}\theta \ev_\theta} \frac1{\sin(\theta)} \pd{}\phi
        -
        \ev_\theta \frac1{\sin(\theta)} \pd{}\phi
        \ev_\phi \pd{}\theta
        +
        \ev_\theta \frac1{\sin(\theta)} \pd{}\phi
        \ev_\theta \frac1{\sin(\theta)} \pd{}\phi
    }
    \intertext{%
        Now there is a bit space left and I will expand the next term.
    }
    &= -
    \sbr{
        \pd{^2}{\theta^2}
        -
        \ev_\phi \sbr{\pd{}\theta \ev_\theta} \frac1{\sin(\theta)} \pd{}\phi
        -
        \ev_\theta \frac1{\sin(\theta)} \sbr{\pd{}\phi \ev_\phi} \pd{}\theta
        -
        \ev_\theta \ev_\phi \frac1{\sin(\theta)} 
        \pd{}\phi \pd{}\theta
        +
        \ev_\theta \frac1{\sin(\theta)} \pd{}\phi
        \ev_\theta \frac1{\sin(\theta)} \pd{}\phi
    }
    \intertext{%
        Here, the first new term has to be evaluated a bit more closely, but
        the second new one is zero because of the orthogonality as well. Now I
        can expand the last term.
    }
    &= -
    \sbr{
        \pd{^2}{\theta^2}
        -
        \ev_\phi \sbr{\pd{}\theta \ev_\theta} \frac1{\sin(\theta)} \pd{}\phi
        -
        \ev_\theta \frac1{\sin(\theta)} \sbr{\pd{}\phi \ev_\phi} \pd{}\theta
        +
        \ev_\theta \frac1{\sin(\theta)^2} \sbr{\pd{}\phi
        \ev_\theta} \pd{}\phi
        +
        \ev_\theta \frac1{\sin(\theta)^2} \sbr{\pd{}\phi
        \ev_\theta} \pd{}\phi
        +
        \frac1{\sin(\theta)^2} \pd{^2}{\phi^2}
    }
    \intertext{%
        The last term can be simplified as well because of the normality. I
        expand the leading minus to get rid of that square bracket.
    }
    &= -
    \pd{^2}{\theta^2}
    +
    \ev_\phi \sbr{\pd{}\theta \ev_\theta} \frac1{\sin(\theta)} \pd{}\phi
    +
    \ev_\theta \frac1{\sin(\theta)} \sbr{\pd{}\phi \ev_\phi} \pd{}\theta
    -
    \ev_\theta \frac1{\sin(\theta)} \sbr{\pd{}\phi
    \ev_\theta} \frac1{\sin(\theta)} \pd{}\phi
    -
    \frac1{\sin(\theta)^2} \pd{^2}{\phi^2}
    \intertext{%
        I computed $\partial \ev_\theta / \partial \theta$ and it turns out to
        be $-\ev_r$. This is orthogonal to $\ev_\phi$. The second term is zero,
        therefore. The derivative $\partial \ev_\theta / \partial \phi$ gives
        $-\cos(\theta) \ev_\phi$ which is orthogonal to $\ev_\theta$. So the
        fourth term drops as well.
    }
    &= - \pd{^2}{\theta^2}
    + \ev_\theta \frac1{\sin(\theta)} \sbr{\pd{}\phi \ev_\phi} \pd{}\theta
    - \frac1{\sin(\theta)^2} \pd{^2}{\phi^2}
    \intertext{%
        The derivative $\partial \ev_\phi / \partial \phi$ gives something that
        is in the $z$-plane and not orthogonal to $\ev_\theta$. The scalar
        product will give a contribution of $-\cos(\theta)$ here.
    }
    &= - \pd{^2}{\theta^2}
    - \cot(\theta) \pd{}\theta
    - \frac1{\sin(\theta)^2} \pd{^2}{\phi^2}
    \intertext{%
        Using the product rule in reverse, this can be written more compactly.
    }
    &= - \sbr{
        \pd{^2}{\theta^2}
        + \cot(\theta) \pd{}\theta
    }
    - \frac1{\sin(\theta)^2} \pd{^2}{\phi^2}
    \intertext{%
        I insert a one into the first summand and expand the cotangent.
    }
    &= - \sbr{
        \frac{1}{\sin(\theta)} \sin(\theta) \pd{^2}{\theta^2}
        + \frac{\cos(\theta)}{\sin(\theta)} \pd{}\theta
    }
    - \frac1{\sin(\theta)^2} \pd{^2}{\phi^2}
    \intertext{%
        Then I factor out the cosecant.
    }
    &= - \frac{1}{\sin(\theta)} \sbr{
        \sin(\theta) \pd{^2}{\theta^2}
        + \cos(\theta) \pd{}\theta
    }
    - \frac1{\sin(\theta)^2} \pd{^2}{\phi^2}
    \intertext{%
        Write the cosine as the derivative of the sine.
    }
    &= - \frac{1}{\sin(\theta)} \sbr{
        \sin(\theta) \pd{^2}{\theta^2}
        + \pd{\sin(\theta)}\theta \pd{}\theta
    }
    - \frac1{\sin(\theta)^2} \pd{^2}{\phi^2} \\
    \intertext{%
        Then I factor out the last derivative.
    }
    &= - \frac{1}{\sin(\theta)} \sbr{
        \sin(\theta) \pd{}{\theta}
        + \pd{}\theta \sin(\theta)
    } \pd{}\theta
    - \frac1{\sin(\theta)^2} \pd{^2}{\phi^2}
    \intertext{%
        Now reverse the product rule.
    }
    &= - \frac{1}{\sin(\theta)} \pd{}{\theta} \sin(\theta) \pd{}\theta
    - \frac1{\sin(\theta)^2} \pd{^2}{\phi^2}
\end{align*}
That is the expression given on the problem set.

Therefore the Laplacian can be written as
\[
    \laplace = \frac{1}{r^2} \pd{}r r^2 \pd{}r - \frac{\vec L^2}{r^2}.
\]
\end{landscape}

\section{Constrain to unit sphere}

\section{Laplacian in two dimensions}

\section{Degeneracy}


\end{document}

% vim: spell spelllang=en tw=79
